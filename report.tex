% Relatório referente ao projeto da disciplina Processamento Digital de Imagens
% do 1o semestre de 2014, ministrada pelo professor Tsang Ing Ren.


\documentclass[a4paper,twocolumn]{article}

% Pacotes necessários
\usepackage{times}
\usepackage[utf8]{inputenc}
\usepackage[english,portuguese]{babel}
\usepackage[a4paper,margin=2cm,columnsep=1cm]{geometry}
\usepackage{authblk}
\usepackage{titlesec}
\usepackage[pdftex]{graphicx}
\usepackage{mathtools}

\DeclareUnicodeCharacter{00A0}{~}
\begin{document}

\graphicspath{{images/}}
\renewcommand{\abstractname}{\normalsize\bfseries\filcenter ABSTRACT}
\titleformat*{\section}{\normalsize\bfseries\filcenter}
\titleformat*{\subsection}{\small\bfseries\filcenter}
\renewcommand{\refname}{\normalsize\bfseries\filcenter REFERENCES}
\renewcommand{\figurename}{\small Figure}
\newcommand{\figureref}[1]{\textit{Fig. \ref{fig:#1}}}
\newcommand{\equationref}[1]{\textit{Eq. \ref{eq:#1}}}


\title{\textbf{Detecção de objetos baseada em LBP utilizando características de textura e borda}}
\author{\textit{Leonardo Brito e Renan Vieira}}
\affil{Centro de Informática, Universidade Federal de Pernambuco\\
Recife, PE, Brasil -- www.cin.ufpe.br\\
\small\texttt{\{lmpb,srfv\}@cin.ufpe.br}}
\date{Junho, 2014}

\maketitle


\begin{abstract}
    \begin{itshape}
      This paper intends to replicate the experiments regarding two novel extraction feature techniques used for object recognition. Both are based on the well-established LBP and LTP extraction feature algorithms. By adding border information to LBP and LTP, these new techniques overcome some of the major limitations of both algorithms. Features are tested on the UIUC Car image dataset.\\ 

    \noindent\textbf{keywords}: Object recognition, local binary pattern, local ternary pattern, feature extraction, texture, border
    \end{itshape}
\end{abstract}


\section{INTRODUÇÃO}

\section{EXPERIMENTAÇÃO}
\label{rule_based}

Por conta de contingências de tempo, não foi possível replicar perfeitamente a metodologia utilizada em \cite{satpathy}. O conjunto UIUC Car possui 170 imagens de teste com escala igual e 108 imagens com escala variável: escolhemos implementar os testes utilizando apenas o primeiro conjunto, de escala única, por este ser de implementação mais simples. As imagens tem uma ordem de grandeza de 1000 janelas, e cada janela leva cerca de 0,67s para a execução do LBP/LTP e cerca de 60s para a execução do DRLTP. A correta execução das técnicas de extração de características em todas as janelas de uma imagem e em todas as 170 imagens do conjunto levaria, portanto, várias horas, impossibilitando o experimento.

Algumas imagens do conjunto tiveram de ser inutilizadas por conta da posição das janelas onde encontravam-se os carros. Carros com oclusão parcial, na fronteira da imagem, resultam em janelas com valores além das dimensões da imagem. Descontados esses casos, obtemos um total de 71 imagens para teste. Cada imagem possui ao menos um carro, portanto foram testadas 71 janelas com resultado positivo esperado. Devido à já mencionada restrição de tempo, apenas cerca de 20 janelas negativas (sem carro) foram testadas para falsos-positivos.

Seguindo a técnica descrita acima, obtemos os seguintes resultados:

\begin{table}[h]
      \begin{tabular}{llllll}
            & VP  & FP    & VN    & FN  & ACERTO \\
      DRLBP & 100 & 22,73 & 77,27 & 0   & 88,64  \\
      LBP   & 0   & 81,82 & 81,82 & 100 & 40,9   \\
      DRLTP & 100 & -     & -     & 0   & 100    \\
      LTP   & 83  & -     & -     & 17  & 83    
      \end{tabular}
\end{table}

Onde VP = verdadeiro positivo, FP = falso positivo, VN = verdadeiro negativo, FN = falso negativo e ACERTO = taxa de acerto. Todos os valores são porcentagens.

Mesmo dadas as limitações da metodologia utilizada, observamos que os resultados são bastante consistentes com os valores esperados levando em conta experimentos semelhantes \cite{satpathy}, \cite{khoo}. O destaque dos resultados é o bom desempenho do LTP relativo ao LBP, já esperado\cite{khoo}, e o bom desempenho do DRLBP e DRLTP em relação às técnicas tradicionais correspondentes.


%IMG:
%\begin{figure}[h]
%    \label{fig:formant_synthesizer}
%    \centering
%    \includegraphics[scale=0.35]{formant-synthesizer}
%    \caption{\textit{Speech synthesizer based on a formant synthesis model.}}
%\end{figure}

%EQUACAO
%\begin{equation}
%    \label{eq:target_cost}
%    C^t(t_i,u_i) = \displaystyle\sum_{j=1}^{p} w^t_j C^t_j(t_i,u_i)
%\end{equation}


\section{CONCLUSÃO}
Embora a metodologia de detecção não seja a mais adequada, ela aponta indícios de que a abordagem do DRLBP e DRLTP melhoram a detecção de objetos em uma imagem. Há uma grande ressalva: ao utilizarmos apenas um conjunto bastante específico de imagens (carros), a robustez dos resultados não é tão boa quanto o teste feito com vários conjuntos diferentes. 

Enquanto o uso de LBP como extrator proporcionou a pior taxa de detecção de janelas positivas (0\%), o DRLBP proporcionou uma detecção de 100\% das janelas positiva. Isto pode indicar que incluir informação de contraste na poderação do histograma melhora a caracterização de um objeto. O que confirma a hipótese de Satpathy, que um objeto é bem representado usando-se informações de textura e borda.

 


% Referências
\begin{thebibliography}{5}
    \bibitem{satpathy}
      Amit Satpathy, Xudong Jiang, e How-Lung Eng,
      "LBP-Based Edge-Texture Features for Object Recognition",
      in \emph{"IEEE Transactions on Image Processing"},
      v.23 n.5,
      p1953,
      Maio 2014.

     \bibitem{ojala_1996}
     Timo Ojala, Matti Pietikäinen e Topi Mäenpää,
     \emph{"Gray Scale and Rotation Invariant Texture Classification with Local Binary Patterns"}
     1996.
     
     \bibitem{ojala_2002}
     Timo Ojala,Topi Mäenpää,
     "Multiresolution Gray-Scale and Rotation Invariant Texture Classification with Local Binary Patterns",
     in \emph{"IEEE Transactions on Pattern Analysis and Machine Intelligence"},
     v.24 n.7,
     p 971,
     Julho 2002.

     \bibitem{khoo}
     Taha H. Rassem e Bee Ee Khoo,
     \emph{"Completed Local Ternary Pattern for Rotation Invariant Texture Classification"},
     2014.
     
      
     \bibitem{han}
     Xian-Hua han, Gang Xu e Yen-Wei Chen,
     "Robust Local Ternary Patterns for Texture Categorization",
     \emph{"2013 6th International Conference on Biomedical Engineering and Informatics (BMEI 2013)"},
     2013.
     
     
     
%    \bibitem{fant}
%        G. Fant,
%        ``Acoustic Theory of Speech Production".
%        \emph{The Hague: Mouton \& Co.}, 1960;
%        \emph{Walter de Gruyter}, 1970.
\end{thebibliography}
\end{document}